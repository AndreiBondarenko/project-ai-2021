%%
%% This is file `beamerthemeuantwerpenuserguide.tex',
%% generated with the docstrip utility.
%%
%% The original source files were:
%%
%% uantwerpendocs.dtx  (with options: `bmrug')
%% 
%% This is a generated file.
%% 
%% Copyright (C) 2013-2021  by Walter Daems <walter.daems@uantwerpen.be>
%% 
%% This work may be distributed and/or modified under the conditions of
%% the LaTeX Project Public License, either version 1.3 of this license
%% or (at your option) any later version.  The latest version of this
%% license is in:
%% 
%%    http://www.latex-project.org/lppl.txt
%% 
%% and version 1.3 or later is part of all distributions of LaTeX version
%% 2005/12/01 or later.
%% 
%% This work has the LPPL maintenance status `maintained'.
%% 
%% The Current Maintainer of this work is Walter Daems.
%% 


\documentclass[aspectratio=1610]{beamer}
\usetheme[we]{uantwerpen}
\usepackage[english]{babel}

\usepackage{metalogo}
\usepackage{kantlipsum}
\usepackage{pgfplots}
\usepackage{booktabs}

\newcommand*\command[1]{{\tt \textbackslash #1}}
\NewEnviron{codesnippet}[1][0.8\textwidth]{
  \scriptsize
  \qquad\framebox[#1][l]{\texttt{
      \setlength\textwidth{#1}
      \begin{minipage}{0.9\textwidth}
        \BODY
      \end{minipage}
    }
  }
}
\newcommand*\ind[1][2ex]{\hspace*{#1}}
\newcommand*\bframe[1][]{\command{begin}\{#1frame\}}
\newcommand*\eframe[1][]{\command{end}\{#1frame\}}

\title{This is your \texttt{\textbackslash{}title{}}}
\subtitle[my short title]{This is your \texttt{\textbackslash{}subtitle{}}}
\date[my short date]{This is your \texttt{\textbackslash{}date{}}}
\author[my short author]{This is your \texttt{\textbackslash{}author{}}}

\begin{document}

\begin{frame}[negativefill]
  \maketitle
\end{frame}

\begin{frame}
  \maketitle
\end{frame}

\begin{frame}[negative]
  \maketitle
\end{frame}

\begin{frame}
  {What's on the menu?}
  {Bon app\'etit!}
  ~\\
  \tableofcontents
\end{frame}

\section{Loading the theme and theme options}

\begin{frame}[negative]
  \sectionpage
\end{frame}

\begin{frame}
  {Using our beamer theme}

  The \texttt{uantwerpen} beamer theme is distributed by CTAN. It
  should be installed with your \TeX{} distribution by default.

  You can load the package in your preamble as:

  \begin{codesnippet}
    \command{documentclass}[aspectratio=1610]\{beamer\}\\
    \command{usetheme}[theme-options-go-here]\{uantwerpen\}\\
    \command{usepackage}[english]\{babel\}
  \end{codesnippet}

  \heading{Remarks}
  \begin{itemize}
  \item The theme behaves well for all supported beamer aspect
    ratios.
  \item $16\times 10$ is currently your best option to maximize your
    slide size given a modern LCD-projector!
  \item Only \texttt{dutch} and \texttt{english} are supported.
    The correct logoware is selected using the babel language option!
  \end{itemize}
\end{frame}

\begin{l3qframe}[t,rqgraphic={
    \begin{minipage}{0.35\textwidth}
      \footnotesize
      \begin{center}
        \begin{tabular}{c}
          \toprule
          \bfseries option\\
          \midrule
          \tt be  \\
          \tt fbd \\
          \tt ggw \\
          \tt lw  \\
          \tt ow  \\
          \tt re  \\
          \tt sw  \\
          \tt ti  \\
          \tt we  \\
          \tt iob \\
          \bottomrule
        \end{tabular}
      \end{center}
    \end{minipage}
  }]
  {Theme options}{}
  The following theme options may come in handy:
  \begin{description}
  \item[\tt X] to invoke your own faculty's colors and logos,
    with X one of the options in the table on the right\\
    (I assume you know your faculty abbreviation)
  \item[\tt nofonts] disables font loading, so you can load your own fonts
  \item[\tt rgb]  to select RGB color encoding (best for projecting, default)
  \item[\tt cmyk] to select CMYK color encoding  (best for printing)
  \item[\tt neutralcolors] to disable your faculty colors and use the
    standard UAntwerpen colors
  \end{description}
  \medskip

  The defaults of the theme are:
  \begin{itemize}
  \item no faculty option  ($\Rightarrow$ standard UAntwerpen logos)
  \item \texttt{rgb}
  \end{itemize}

\end{l3qframe}

\begin{frame}[t]
  {A note on the compiler you are using}
  {Fonts galore}

  Depending on the type of compiler you use, a different font scheme
  is loaded:
  \begin{description}
  \item[\LaTeX] --- ancient compiler - no support\\
    Don't use!
  \item[pdf\LaTeX] --- very old school compiler - no OTF/TTF support\\
    Computer Modern Sans Serif is used as font
  \item[\XeLaTeX] --- old school compiler - the first to have OTF/TTF
    support\\
    Calibri is used as font (with cmbright math fonts)
  \item[\LuaLaTeX] --- current compiler - your best option\\
    Calibri is used as font (with cmbright math fonts)
  \end{description}

\end{frame}

\section{Title slides}

\begin{frame}[negative]
  \sectionpage
\end{frame}

\begin{frame}[t]
  {Title slides}
  \heading{Contents} ---
  The contents of the title slide can be set in your preamble
  using the classical \LaTeX{} commands:
  \begin{itemize}
  \item \command{title\{\}}
  \item \command{subtitle}\{\}
  \item \command{author\{\}}
  \item \command{date\{\}}
  \end{itemize}
  We don't specify an institute, as the logos do so.

  \heading{Code} --- The titleslide is easily typeset as:\\
  \begin{codesnippet}
    \bframe[][option]\\
    \ind\command{maketitle}\\
    \eframe
  \end{codesnippet}

  with no option, or one of: \texttt{normal}, \texttt{negative},
  \texttt{negativefill}.
\end{frame}

\section{Regular slides}

\subsection{Bare}

\begin{frame}[negative]
  \sectionpage
\end{frame}

\begin{frame}[negative]
  \subsectionpage
\end{frame}

\begin{frame}[t]
  {Frame titles}
  {And subtitles}
  \heading{How to specify them!}\\
  Titles are specified using \command{frametitle\{\}} or
  \command{framesubtitle\{\}} or even easier as first and second argument
  to the \texttt{frame} environment (or their derived versions).

  \begin{codesnippet}
    \bframe[][options go here, comma separated]\\
    \ind\{first argument\}\\
    \ind\{second argument\}\\
    \ind frame contents \\
    \eframe
  \end{codesnippet}
  \bigskip

  \heading{Don't need them?}\\
  If you need a slide without titles: just don't specify them!

  \heading{Want to get rid of the footer as well?}\\
  Specify the \texttt{plain} option to the frame.

\end{frame}

\begin{frame}[t]
  {The canvas}

  \heading{Flavors} --- The canvas of the slide has four flavors:
  \begin{description}
  \item[normal]
    the ordinary white background slide
  \item[negativefill]
    a reverse video slide on a background (in maincolor) without white
    margins
  \item[negative]
    a reverse video slide on a bakcground (in maincolor) with white
    margins
  \item[graphicfill]
    a graphic canvas without white margins (graphic can be photo,
    graph, \ldots)
  \item[graphic] a graphic canvas with white margins (graphic can be
    anything)
  \end{description}
  They are specified as options to the frame environment (or its
  derivatives).

  \heading{Remarks}
  \begin{itemize}
  \item You don't need to specify the normal canvas, it is the
    default.
  \item In handout mode the negative and negativefill canvas will be
    typeset as normal canvas (to allow for easy printing).
  \item The graphic option has many variants that we will discuss later.
  \end{itemize}
\end{frame}

\begin{frame}[negativefill]
  {A negativefill slide}
  {with a subtle subtitle}
  \kant[1]
\end{frame}

\begin{frame}[negative,t]
  {A negative slide}
  {with a subtle subtitle}
  Don't do this! Don't use a title and subtitle, nor straight text
  but put material on this slide that does not touch or cross the
  edges of the background!\\
  E.g., the graph on the bottom right
  \place[anchor=south east] at (0.9,0.1) {
    \begin{tikzpicture}[white]
      \begin{axis}
        [width=4cm,height=5cm,grid=both,font=\footnotesize]
        \addplot[white] {x^2};
      \end{axis}
    \end{tikzpicture}
  }
  The best advice is not to reserve this canvas for title frames and
  intermission slides.
\end{frame}

\begin{frame}[t]
  {Vertical alignment}

  You can easily specify the vertical alignment of your frame
  contents, using the options
  \begin{description}
  \item[\texttt{t}] for top
  \item[\texttt{c}] for center
  \item[\texttt{b}] for bottom
  \end{description}

  Example:\\[1ex]
  \begin{codesnippet}
    \bframe[][t]\\
    \ind\{Title\}\\
    \ind\{Subtitle\}\\
    \ind frame contents\\
    \eframe
  \end{codesnippet}

\end{frame}

\begin{frame}[t]
  {Colors}
  {What a wonderful world}

  \heading{Standard colors}\\
  You can select the theme colors using:\\
  \textcolor{maincolor}{Main color}:
  can be specified as \LaTeX-color \emph{maincolor}\\
  \textcolor{sidecolor}{Side color}:
  can be specified as \LaTeX-color \emph{sidecolor}\\
  \textcolor{basecolor}{Base color}:
  can be specified as \LaTeX-color
  \emph{basecolor}

  \heading{Advice}\\
  \begin{itemize}
  \item Stick to the standard colors or use grayscale tints!
  \item Only use color when functional (e.g. in graphs)
  \end{itemize}

  \heading{Alert}
  You can use \alert{\command{alert{}}} to grab the attention of the
  user. It will typeset your content in \alert{red}.

  \heading{Not taking any advice?}\\
  You can fiddle with the colors, but do so in smart way.
  Use \command{setbeamercolor}. You can see how it is used in the
  \texttt{beamercolorthemeuantwerpen.sty} file.
\end{frame}

\begin{frame}[t]
  {Logo demo}

  The logo's can be used as follows (but you should not need them):
  \begin{itemize}
  \item \command{includegraphics[width=3cm]\{\command{logopos}\}}\\
    \begin{center}
      \begin{tikzpicture}
        \draw[fill,white] (0,0) rectangle node
        {\includegraphics[width=3cm]{\logopos}} (4,1.25);
      \end{tikzpicture}
    \end{center}
  \item \command{includegraphics[width=3cm]\{\command{logoneg}\}}\\
    \begin{center}
      \begin{tikzpicture}
        \draw[fill] (0,0) rectangle node
        {\includegraphics[width=3cm]{\logoneg}} (4,1.25);
      \end{tikzpicture}
    \end{center}
  \item \command{includegraphics[width=3cm]\{\command{logomonowhite}\}}\\
    \begin{center}
      \begin{tikzpicture}
        \draw[fill] (0,0) rectangle node
        {\includegraphics[width=3cm]{\logomonowhite}} (4,1.25);
      \end{tikzpicture}
    \end{center}
  \end{itemize}
\end{frame}

\subsection{With graphic eye candy}

\begin{frame}[negative]
  \subsectionpage
\end{frame}

\begin{frame}[t]
  {Basic idea}

  \heading{The idea} --- put support material (photo or graph) on the
  slide next to the frame content.
  This is done by
  \begin{itemize}
  \item specifying an appropriate canvas
  \item using an appropriate frame derivative
  \end{itemize}

  \heading{Good combinations}
  \begin{center}\small
    \begin{tabular}{ccc}
      \toprule
      \bfseries frame option & \bfseries graphic on frame
      & \bfseries corresponding frame derivative\\
      \midrule
      lqgraphic & left quarter   & r3qframe \\
      rqgraphic & right quarter & l3qframe \\
      lhgraphic & left half      & rhframe \\
      rhgraphic & right half    & lhframe \\
      thgraphic & top half       & bhframe \\
      bhgraphic & bottom half   & thframe\\
      \bottomrule
    \end{tabular}
  \end{center}

\end{frame}

\begin{l3qframe}[rqgraphic]
  {Some Random Title}
  {Please, adapt!}

  \small
  The slide has been typeset as:\\
  \begin{codesnippet}[\textwidth]
    \bframe[l3q][rqgraphic=\{<load photo here>\}]\\
    \ind\{Title\}\\
    \ind\{Subtitle\}\\
    \ind frame contents\\
    \eframe[l3q]
  \end{codesnippet}
  \smallskip

  \subheading{Remarks}
  \begin{itemize}
  \item Load photo as:\\
    \command{includegraphics[width=0.25\command{paperwidth}, min height=\command{textheight}]
      \{image.jpg\}}
  \item If \texttt{rqgraphic} is specified without argument, a standard photo
    is loaded.
  \item Note: you can avoid rounding the corner using the
    frame option \texttt{noround} (may be useful when not using a photo)
  \end{itemize}
\end{l3qframe}

\begin{r3qframe}[lqgraphic,t]
  {Some Random Title}
  {Please, adapt!}

  \small
  The slide has been typeset as:\\
  \begin{codesnippet}[\textwidth]
    \bframe[r3q][lqgraphic=\{<load photo here>\}]\\
    \ind\{Title\}\\
    \ind\{Subtitle\}\\
    \ind frame contents\\
    \eframe[r3q]
  \end{codesnippet}
  \smallskip

  \subheading{Remarks}
  \begin{itemize}
  \item   Load photo as:\\
    \command{includegraphics[width=0.25\command{paperwidth}, min height=\command{textheight}]
      \{image.jpg\}}
  \item If \texttt{rqgraphic} is specified without argument, a standard photo
    is loaded.
  \item Note: you can avoid rounding the corner using the
    frame option \texttt{noround} (may be useful when not using a photo)
  \end{itemize}
\end{r3qframe}

\begin{lhframe}[rhgraphic]
  {Some Random Title}

  \small
  The slide has been typeset as:\\
  \begin{codesnippet}[\textwidth]
    \bframe[lh][rhgraphic=\{<load photo here>\}]\\
    \ind\{Title\}\\
    \ind\{Subtitle\}\\
    \ind frame contents\\
    \eframe[lh]
  \end{codesnippet}
  \smallskip

  \subheading{Remarks}
  \begin{itemize}
  \item   Load photo as:\\
    \command{includegraphics[min
      width=0.5\command{paperwidth}, min height=\command{textheight}]
      \{image.jpg\}}
  \item If \texttt{rqgraphic} is specified without argument, a standard photo
    is loaded.
  \item Note: avoid rounding the corner using \texttt{noround}
  \end{itemize}
\end{lhframe}

\begin{rhframe}[lhgraphic,t]
  {Some Random Title}

  \small
  The slide has been typeset as:\\
  \begin{codesnippet}[\textwidth]
    \bframe[rh][lhgraphic=\{<load photo here>\}]\\
    \ind\{Title\}\\
    \ind\{Subtitle\}\\
    \ind frame contents\\
    \eframe[rh]
  \end{codesnippet}
  \smallskip

  \subheading{Remarks}
  \begin{itemize}
  \item   Load photo as:\\
    \command{includegraphics[min
      width=0.5\command{paperwidth}, min height=\command{textheight}]
      \{image.jpg\}}
  \item If \texttt{rqgraphic} is specified without argument, a standard photo
    is loaded.
  \item Note: avoid rounding the corner using \texttt{noround}
  \end{itemize}
\end{rhframe}

\begin{bhframe}[thgraphic,t]
  {Some Random Title}{Please, adapt!}
  \small
  The slide has been typeset as:
  \begin{codesnippet}[\textwidth]
    \bframe[bh][thgraphic=\{<load photo here>\}]\\
    \ind\{Title\}\\
    \ind\{Subtitle\}\\
    \ind frame contents\\
    \eframe[bh]
  \end{codesnippet}
\end{bhframe}

\begin{thframe}[bhgraphic,t]
  {Some Random Title}
  {Please, adapt!}
  \small
  The slide has been typeset as:
  \begin{codesnippet}[\textwidth]
    \bframe[th][bhgraphic=\{<load photo here>\}]\\
    \ind\{Title\}\\
    \ind\{Subtitle\}\\
    \ind frame contents\\
    \eframe[th]
  \end{codesnippet}
\end{thframe}

\section{Intermission slides}

\begin{frame}[negative]
  \sectionpage
\end{frame}

\subsection{Standard section slides}

\begin{frame}[negative]
  \subsectionpage
\end{frame}

\begin{frame}[t]
  {Basic idea}

  You can select any of the canvas templates: negative, negativefill,
  graphic, graphicfill and normal, and combine it with:
  \begin{description}
  \item[\command{sectionpage}]
  \item[\command{subsectionpage}]
  \end{description}

  Example:\\[1ex]
  \begin{codesnippet}[0.8\textwidth]
    \bframe[][negative]\\
    \ind \command{sectionpage}\\
    \eframe\\~\\
    \bframe[][negative]\\
    \ind \command{subsectionpage}\\
    \eframe
  \end{codesnippet}\\[1ex]
  (this how the previous two slide were generated)
\end{frame}

\subsection{Custom intermission slides}

\begin{frame}[negative]
  \subsectionpage
\end{frame}

\begin{frame}[t]
  {Basic idea}

  If you are a fan of graphic material in slideware, you can use a
  \texttt{graphic} or \texttt{graphicfill} canvas in combination with
  small snippets you put on the slide using\\[1ex]
  \qquad\command{place[node options] at (x,y) \{<material>\}}\\[1ex]
  with
  \begin{itemize}
  \item $(x,y)=(0,0)$ the bottom left of the slide and $(1,1)$ the top
    right of the slide
  \item \texttt{node options} any options you want to hand over to the
    tikz node that is used to position the material.
  \end{itemize}
  \medskip

  E.g., the next slide was created using:\\[1ex]
  \begin{codesnippet}[0.8\textwidth]
    \bframe[][graphic,t]\\
    \ind\command{place} at (0.5,0.5) \{\command{uantwerpenicon[scale=0.8]}\}\\
    \eframe
  \end{codesnippet}
  \medskip

  You will have to run \LaTeX{} twice in order for the position to be correct!
\end{frame}

\begin{frame}[graphic,t]
  \place at (0.5,0.5) {\uantwerpenicon[scale=0.8]}
\end{frame}

\begin{frame}[t]
  {Shading photographs}

  Often you need make a photograph a little bit more opaque in order
  for the (white) text to readable.

  To this end you can use the following command\\[1ex]
  \qquad\command{darken[s]\{<material>\}}\\[1ex]
  with $s$ a value between 0 and 1 to specify the level of darkening.

  E.g., the next slide was created using:
  \begin{codesnippet}[0.9\textwidth]
    \bframe[][graphicfill=\{\command{darken}%
    [0.25]\{\command{includegraphics}%
    [scale=0.1,min width=\command{paperwidth},min height=\command{paperheight}]%
    \{Images/uantwerpen-09.jpg\}\},t]\\
    \command{place} [anchor=north east] at (0.95,0.5) \{\\
    \ind\command{uantwerpencallout}\{3\}\{2.5\}\{\\
    \ind\ind\command{bfseries} You cannot be serious!\textbackslash\textbackslash[1ex]\\
    \ind\ind John McEnroe\}\\
    \}\\
    \eframe
  \end{codesnippet}
\end{frame}

\begin{frame}[graphicfill={\darken[0.25]%
    {\includegraphics[scale=0.1,min width=\paperwidth,min height=\paperheight]%
      {Images/uantwerpen-09.jpg}}},t]
  \place [anchor=north east] at (0.95,0.5) {
    \uantwerpencallout{3}{2.5}{
      \bfseries You cannot be serious!\\[1ex]
      John McEnroe}
  }
\end{frame}

\begin{frame}[t]
  {Shading photographs locally}

  You can also choose to shade the local backdrop of the material
  you put on the side. If you like the \command{uantwerpencallout}
  command of the previous slide, you can give it an optional argument
  to set the fill opacity of the backdrop behind the text.

  E.g., the next slide was created using:
  \begin{codesnippet}[0.9\textwidth]
    \bframe[][graphic,t]\\
    \ind\command{place} [anchor=north west] at (0.1,0.9) \{\\
    \ind\ind\command{uantwerpencallout}[fill opacity=0.5]\{4.5\}\{2.5\}\{\\
    \ind\ind\ind\command{bfseries} I'll be back!\textbackslash\textbackslash[1ex]\\
    \ind\ind\ind Arnold Schwarzenegger\}\\
    \}
    \eframe
  \end{codesnippet}
  \medskip

  The \command{uantwerpencallout} command has te following syntax:\\[1ex]
  \qquad\command{uantwerpencallout[options]\{width\}\{height\}\{contents\}}
  \medskip

  Remember to run \LaTeX{} twice in order for the position of the
  callout to be correct!
\end{frame}

\begin{frame}[graphic,t]
  \place [anchor=north west] at (0.1,0.9) {
    \uantwerpencallout[fill opacity=0.5]{4.5}{2.5}{
      \bfseries I'll be back!\\[1ex]
      Arnold Schwarzenegger}
  }
\end{frame}

\section{Demo of some inner theme parts}

\begin{frame}[negative]
  \sectionpage
\end{frame}

\begin{frame}[t]
  {Headings}
  {on two levels}
  You can set headings on two levels, by using
  \begin{itemize}
  \item\command{heading\{\}}
  \item\command{subheading\{\}}
  \end{itemize}
  \heading{Main subject}~\\
  blabla
  \subheading{Subtopic 1}~\\
  blabla
  \subheading{Subtopic 2}~\\
  blabla
  \heading{Next main subject}~\\
  blabla
\end{frame}

\newcommand\listdemo[1][itemize]{
  \begin{#1}
  \item First level
    \begin{#1}
    \item Second level
      \begin{#1}
      \item Third level
      \end{#1}
    \end{#1}
  \end{#1}
}

\begin{frame}[t]
  {Itemize / enumerations}
  {Different styles}
  Usage:\\
  Set them (in your preamble) using: \command{setbeamertemplate\{itemize items\}[triangle]}\\
  The default of the uantwerpen beamer template is set to \texttt{square} to be
  similar to our logo.
  \medskip

  \begin{columns}
    \column[T]{0.3\textwidth}
    \alert{default}
    \setbeamertemplate{itemize items}[default]
    \listdemo
    \column[T]{0.3\textwidth}
    \alert{triangle}
    \setbeamertemplate{itemize items}[triangle]
    \listdemo
    \column[T]{0.3\textwidth}
    \alert{circle}
    \setbeamertemplate{itemize items}[circle]
    \listdemo
  \end{columns}~\\
  \bigskip

  \begin{columns}
    \column[T]{0.3\textwidth}
    \alert{ball}
    \setbeamertemplate{itemize items}[ball]
    \listdemo
    \column[T]{0.3\textwidth}
    \alert{square}
    \setbeamertemplate{itemize items}[square]
    \listdemo
    \column[T]{0.3\textwidth}
    \alert{enumeration}
    \setbeamertemplate{itemize items}[circle]
    \listdemo[enumerate]
  \end{columns}
\end{frame}

\begin{frame}[t]
  {Block material}
  {Nothing special}

  \vfill
  \begin{block}{This is a block}
    Lorem ipsum dolor sit amet, consectetur adipiscing elit. Morbi ac
    arcu est, vel posuere velit. In congue erat vel lorem ornare pretium.
  \end{block}
  \vfill
  \begin{exampleblock}{This is an example block}
    Lorem ipsum dolor sit amet, consectetur adipiscing elit. Morbi ac
    arcu est, vel posuere velit. In congue erat vel lorem ornare pretium.
  \end{exampleblock}
  \vfill
  \begin{alertblock}{This is an alert block}
    Lorem ipsum dolor sit amet, consectetur adipiscing elit. Morbi ac
    arcu est, vel posuere velit. In congue erat vel lorem ornare pretium.
  \end{alertblock}
  \vfill
\end{frame}
\section{Advanced material}

\begin{frame}[negative]
  \sectionpage
\end{frame}

\begin{frame}[t]
  {Customizing the template}

  If you want to override logos/colors to create a template for your
  own research group or department:
  renew the commands below. The redefinitions must be put just after
  the \command{begin\{document\}} statement.\\
  The construction below allows you to put the stuff in a style file
  that you must load after (!) the \command{usetheme} command.\\[1ex]
  \begin{codesnippet}[0.95\textwidth]
    \command{AtBeginDocument\{}\\
    \ind\command{renewcommand*}\command{logopos\{your-pos-logo-filename-here\}}\\
    \ind\command{renewcommand*}\command{logoneg\{your-negative-logo-filename-here\}}\\
    \ind\command{renewcommand*}\command{logomonowhite\{your-mono-logo-filename-here\}}\\
    \ind\command{renewcommand*}\command{iconfile\{your-icon-filename-here\}}\\
    \ind\command{colorlet\{maincolor\}\{your-favorite-color\}}\\
    \ind\command{colorlet\{sidecolor\}\{your-secondfavorite-color\}}\\
    \ind\command{colorlet\{basecolor\}\{some-solid-color\}}\\
    \}
  \end{codesnippet}~\\[1ex]
\end{frame}

\begin{frame}
  {Some genuine advice}
  \small

  Dear user,

  Candy slides are nice for PR, but bad for conveying a message.

  Beamer is a tool to typeset technical presentations. Need more
  animation and eye candy? Use other tools. They are much easier in
  ruining your audiences' day.

  Effective slides are simple slides.\\
  Go by cognitive consonance, flirt with cognitive dissonance, but
  stay away from cognitive cacophony.
  This presentation is i.m.ho. a bad one, as it shows the
  cacophony of possibilities. But hey, you were the one asking for it.

  A lot of effort has gone into this template. I hope you like it.
  If you have good suggestions, e-mail me. If you have questions, I
  might even help you.

  Cheers!

  Walter Daems\\
  (March 2021)

\end{frame}

\end{document}
\endinput
%%
%% End of file `beamerthemeuantwerpenuserguide.tex'.
